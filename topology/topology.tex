\documentclass{article}[12pt]

\usepackage{amsmath}
\usepackage{amsthm}

\usepackage{multicol}

\theoremstyle{definition}
\newtheorem{theorem}{Theorem}[section]
\newtheorem{definition}[theorem]{Definition}

\begin{document}
\title{Topology}
\author{Sean Richardson}
\date{\today}
\maketitle

\section{Manifolds}
\subsection{1 dimensional}
\begin{itemize}
    \item The Line ($E^1$)
    \item The Circle ($S^1$)
    \item The Interval ($I$)
\end{itemize}
\subsection{2 dimensional}
\begin{itemize}
    \item Euclidean Plane ($E^2$)
    \item Sphere ($S^2$)
    \item Torus ($T^2$)
    \item Klein Bottle ($K^2$)
    \item Projective Plane ($P^2$)
    \item Disk ($D^2$)

\end{itemize}

\subsection{3 dimensional}
\begin{itemize}
    \item Three dimensional Euclidean Space ($E^3$)
    \item Three Torus ($T^3$)
    \item Solid Ball ($D^3$)
    \item Projetive three-space ($P^3$)
\end{itemize}

\section{Terms}
\begin{definition}[Topology vs. Geometry]~\\
    The \emph{topology} of a surface includes the properties of the surface that do not change when you stretch, twist, bend (without tearing) the surface. Such properties are:
    \begin{itemize}
        \item Number of holes the surface has
        \item \dots
    \end{itemize}
    The \emph{geometry} of a surface indludes all properties that do change when deforming the surface without tearing. Such properties are:
    \begin{itemize}
        \item \dots
    \end{itemize}
\end{definition}

\begin{definition}[Instrinsic vs. Extrinsic]~\\
    The \emph{intrinsic} properties of a surface are those a flatlander would notice while the \emph{extrinsic} properties are those a flatlander would not notice, but a being one dimension higher looking at the surface would. These properties can be topological or geometrical. Note: a manifold does not need to exist in a higher dimensional space. It can exits in and of itself. 
\end{definition}
\begin{definition}[Local vs. Global]~\\
    \emh{Local} proprties can be observed in an (arbitrarily) small region of the manifold while \emph{global} proprties require consideration of the whole manifold.
\end{definition}
\begin{definition}[Homogenous vs. Nonhomogenous Geometries]

\end{definition}
\begin{definition}[Closed vs. Open Manifold]

\end{definition}
\begin{definition}[Orientable vs. Nonorientable]

\end{definition}
\end{document}
