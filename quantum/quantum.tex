\documentclass[11pt]{article}

\usepackage{amsmath}
\usepackage{amsthm}
\usepackage{amssymb}

\usepackage{my_math}

\theoremstyle{definition}
\newtheorem{theorem}{Theorem}[section]
\newtheorem{definition}[theorem]{Definition}

\newcommand{\pder}[2]{\frac{\partial{#1}}{\partial{#2}}}
\newcommand{\pdder}[2]{\frac{\partial^2{#1}}{\partial{#2^2}}}

\setlength\parindent{0pt}
\begin{document}
\title{Quantum}
\author{Sean Richardson}
\date{\today}
\maketitle
\tableofcontents

\section{Historical Remarks}
\subsection{Ultraviolet Catastrophe}
\begin{definition}[Black Body]
    A \emph{black body} is some object that satisfies the following
    conditions:
\begin{itemize}
    \item Absorbs all incoming radiation
    \item Emits the entire spectrum
    \item Is at equilibrium (temperature does not change)
\end{itemize}
We model Electromagnetic radiation with infinite \emph{harmonic
oscillators}.
We model a harmonic oscillator of frequency $\nu$ mathematically with:
\begin{equation}
    \Phi(x,t) = A\cos(\omega t - kx)
    \tag{Harmonic Oscillator}
\end{equation}
Where the wave number $k \equiv \frac{2\pi}{\nu}$, the angular frequency
$\omega \equiv 2\pi \nu$ and we have $\lambda = \frac{c}{\nu}$.
The energy of a harmonic oscillator is given by
\begin{equation}
    \overline{E_{h.o.}} = k_B T
    \label{eq:eo}
\end{equation}
We find that the number density of harmonic oscillators per volume at a
specific frequency $\nu$ is given by:
\begin{equation}
    n(\nu) = \frac{8\pi \nu^2}{c^3}
    \label{eq:nd}
\end{equation}
We can then combine (\ref{eq:eo}) and (\ref{eq:nd}) to find the energy
density $\rho$ at temperature $T$ and frequency $\nu$:
\begin{equation*}
    \rho(\nu,T) = \overline{E_{h.o.}} n(\nu) = \frac{8\pi \nu^2 k_B T}{c^3}
\end{equation*}
However, we then have that the total energy of a black body is given by
\begin{equation*}
    E_{total} = \int_0^\infty \rho(\nu,T)d\nu = \frac{8 \pi k_B T}{c^3}
    \int_0^\infty \nu^2 d\nu = \infty
\end{equation*}
So, the current model predicts infinite energy of a black body. This
clearly incorrect result is called the \emph{ultraviolet catastrophe}. So,
we must change our model.
\end{definition}
\marginpar{black body curve}
Planck solved this in assuming that oscillators of the black body are
quantized. They can only be in energy states $E = nh\nu$, $n \in
\mathbb{Z}_{\geq 0}$ (Einstein later showed that it is the E.M. radiation
itself that is quantized). Under this assumption we find that the energy
density is given by
\begin{equation*}
    \rho(\nu, T) = \frac{8\pi h}{c^3}\frac{\nu^3}{e^{h \nu / (k_B T)}-1}
    \tag{Plank's Formula}
\end{equation*}
Which fits the experimental black body curve. Note: each $h\nu$ ``package''
of energy is called a photon.
\subsection{Atomic Spectrum}
Atomic gases were experimentally found to emit light at only discrete
frequencies.
Bohr's model offered an explanation for this.
Bohr adopted Rutherford's model, of an orbiting electron (this model was
rejected because maxwell's equations predict the electron would emit
radiation making it unstable). Bohr assumes that angular momentum is
quantized. Such that:
$$ \myabs{\mathbf{L}} = \myabs{\mathbf{r} \times \mathbf{p}} = n \hbar $$
Where $n \in \mathbb{N}$. Note $\hbar = \frac{h}{2\pi}$.
Then, we find the energy of an electron must be,
$$ E_n = \frac{-13.6}{n^2} eV $$
So an emitted photon has energy $E_n-E_m$, $n,m\in\mathbb{N}$, which fits
experimental data.
\\
But why is angular momentum quantized? Louis de Broglie answered this
with the assumption that an electron (when thought of as a wave) should
satisfy periodic boundary conditions. Or, $2\pi r = n \lambda$ where $r$ is
the orbiting radius. Next, de Broglie some assumptions:
\begin{itemize}
    \item For any particle, $E = h\nu$ just as in the case of a photon.
    \item For a photon, $E = pc$ from relativity.
\end{itemize}
Then, if we call the wavelength of the photon $\lambda_{dB}$,
$$ \lambda_{dB} = \frac{h}{p} $$
We then extend this from the case of a photon to everything, which works.
Additionally, in combining de Broglie's insights: $\lambda_{dB} =
\frac{h}{p}$ and $2\pi r = n\lambda$, we find that angular momentum is
quantized as Bohr predicted, $\myabs{\mathbf{L}} = n \hbar$.
\subsection{Schr\"odinger's Equation}
To summarize, we find that particles have wave-like properties.
Specifically, $E=h\nu$, and $p = \frac{h}{\lambda_{dB}}$. Or, $E = \hbar
\omega$ and $p = \hbar k$.
So, we assign a wave equation to particles of the form
$$ \Psi(x,t) = e^{i(kx-\omega t)} $$
This is a solution to the following ODE's:
\begin{align*}
    \frac{-\hbar}{2m} \pdder{\Psi(x,t)}{x} &= E\Psi(x,t)
    \tag{ODE 1}\\
    i\hbar\pder{\Psi(x,t)}{t} &= E\Psi(x,t)
    \tag{ODE 2}
\end{align*}
This motivates the following PDE, Schr\"odinger's equation with the
additional generalization that $E = \frac{p^2}{2m} + \mathcal{V}(x)$
\begin{equation}
    \frac{-\hbar}{2m} \pdder{\Psi(x,t)}{x}
    + \mathcal{V}(x)\Psi(x,t)
    = i\hbar\pder{\Psi(x,t)}{t}
    \tag{Schr\"odinger's equation}
\end{equation}
It turns out that the solutions to this equation 
\section{}
\end{document}
