\documentclass[11pt]{article}

\usepackage{amsmath}

\usepackage{enumitem}
\usepackage{commath}
\usepackage{tikz}
\usetikzlibrary{calc}
\usepackage{cancel}
\usepackage{subfigure}
\usepackage{multicol}

\usepackage{graphicx}
\graphicspath{{images/}}

\usepackage{my_notes}
\usepackage{my_math}

\newenvironment{mybox}
{\begin{tcolorbox}[colback=red!5!white,colframe=red!75!black]}
{\end{tcolorbox}}

\newtheorem{proof}{Proof}
\newtheorem{theorem}{Theorem}

\newenvironment{mytbox}
{\begin{tcolorbox}[colback=red!5!white,colframe=red!75!black,title=test]}
{\end{tcolorbox}}

\begin{document}
\title{Discrete}
\author{Sean Richardson}
\date{\today}
\maketitle

\section{Combinatorics}

Combinatorics studies methods of counting things.

\begin{theorem}
    Addition Principle: Consider some task $T$. If $T$ can be accomplished by methods $M_1,M_2,\dots,M_n$ which can each be accomplished in $a_1, a_2, \dots, a_n$ ways, then $T$ can be accomplished in $\sum a_k$ ways.
\end{theorem}
\begin{theorem}
    Multiplication Principle: Consider some task $T$. If $T$ can be broken down into necessary subtasks $t_1,t_2,\dots,t_n$ which can be accomplished in $b_1,b_2,\dots,b_n$ ways, then $T$ can be done in $\prod b_k$ ways.
\end{theorem}
\begin{theorem}
    An arrangment of $n$ objects is called a ``permutaion''. There are $n!$ possible permutations.
\end{theorem}
\begin{theorem}
    An arrangment of $r$ objects out of a collection of $n$ objects is called an ``r-permutation''. This can be done in $P(n,r)= {}_{n}P_r = \frac{n!}{(n-r)!}$ ways.
\end{theorem}
\begin{theorem}
    An r-combination is how many combinations of $r$ objects (ignoring order) you can choose from $n$ objects. There are $C(n,r)={}_{n}C_r=(\begin{smallmatrix}n \\ r \end{smallmatrix}) = \frac{n!}{r!(n-r)!}$ ways.
\end{theorem}
We will now try to determine how many ways $n$ balls can be put in to $r$ boxes. To see this, we use the following trick. Think of balls distributed into boxes in the following structure: $\circ \circ / \circ \circ \circ / \circ / \circ \circ$ where ``$\circ$'' represents balls and the ``/'' symbols divide the balls into categories or boxes. So, the amount of ways to divide $n$ balls into $r$ categories is the amount of ways we can the distribute the dividers among the characters. This is equivalent to ``\textit{characters} choose \textit{dividers}''. There are $n+r-1$ characters and $r-1$ dividers. So, we have $(\begin{smallmatrix} n+r-1 \\ r-1 \end{smallmatrix})$ ways to divide $n$ identical things into $r$ categories. Additionally, we have the following useful equivalency:
\begin{align*}
    \begin{pmatrix} n+r-1 \\ r-1 \end{pmatrix}
    =\frac{(n+r-1)!}{(r-1)!(n+r-1-(r-1))} \\
    =\frac{(n+r-1)!}{(n+r-1-(n))!(n)!}
    = \begin{pmatrix} n+r-1 \\ n \end{pmatrix}
\end{align*}
\begin{theorem}
    The number of ways to distribute $n$ identical balls into $r$ distinct boxes is $(\begin{smallmatrix} n+r-1 \\ r-1 \end{smallmatrix})$ or $(\begin{smallmatrix} n+r-1 \\ n \end{smallmatrix})$ ways.
\end{theorem}
\end{document}
