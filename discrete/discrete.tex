\documentclass[11pt]{article}

\usepackage{amsmath}
\usepackage{amsthm}

\usepackage{enumitem}
\usepackage{commath}
\usepackage{tikz}
\usetikzlibrary{calc}
\usepackage{cancel}
\usepackage{subfigure}
\usepackage{multicol}

\usepackage{graphicx}
\graphicspath{{images/}}

\usepackage{my_notes}
\usepackage{my_math}

\newenvironment{mybox}
{\begin{tcolorbox}[colback=red!5!white,colframe=red!75!black]}
{\end{tcolorbox}}

%
\theoremstyle{definition}
\newtheorem{theorem}{Theorem}[section]
\newtheorem*{proof*}{Proof}
\newtheorem{definition}{Definition}[section]

\newenvironment{mytbox}
{\begin{tcolorbox}[colback=red!5!white,colframe=red!75!black,title=test]}
{\end{tcolorbox}}

\setlength\parindent{0pt}

\begin{document}
\title{Discrete}
\author{Sean Richardson}
\date{\today}
\maketitle

\section{Combinatorics}

Combinatorics studies methods of counting things.

\begin{theorem}
    Addition Principle: Consider some task $T$. If $T$ can be accomplished by methods $M_1,M_2,\dots,M_n$ which can each be accomplished in $a_1, a_2, \dots, a_n$ ways, then $T$ can be accomplished in $\sum a_k$ ways.
\end{theorem}
\begin{theorem}
    Multiplication Principle: Consider some task $T$. If $T$ can be broken down into necessary subtasks $t_1,t_2,\dots,t_n$ which can be accomplished in $b_1,b_2,\dots,b_n$ ways, then $T$ can be done in $\prod b_k$ ways.
\end{theorem}
\begin{theorem}
    An arrangment of $n$ objects is called a ``permutaion''. There are $n!$ possible permutations.
\end{theorem}
\begin{theorem}
    An arrangment of $r$ objects out of a collection of $n$ objects is called an ``r-permutation''. This can be done in $P(n,r)= {}_{n}P_r = \frac{n!}{(n-r)!}$ ways.
\end{theorem}
\begin{theorem}
    An r-combination is how many combinations of $r$ objects (ignoring order) you can choose from $n$ objects. There are $C(n,r)={}_{n}C_r=(\begin{smallmatrix}n \\ r \end{smallmatrix}) = \frac{n!}{r!(n-r)!}$ ways.
\end{theorem}
We will now try to determine how many ways $n$ balls can be put in to $r$ boxes. To see this, we use the following trick. Think of balls distributed into boxes in the following structure: $\circ \circ / \circ \circ \circ / \circ / \circ \circ$ where ``$\circ$'' represents balls and the ``/'' symbols divide the balls into categories or boxes. So, the amount of ways to divide $n$ balls into $r$ categories is the amount of ways we can the distribute the dividers among the characters. This is equivalent to ``\textit{characters} choose \textit{dividers}''. There are $n+r-1$ characters and $r-1$ dividers. So, we have $(\begin{smallmatrix} n+r-1 \\ r-1 \end{smallmatrix})$ ways to divide $n$ identical things into $r$ categories. Additionally, we have the following useful equivalency:
\begin{align*}
    \begin{pmatrix} n+r-1 \\ r-1 \end{pmatrix}
    =\frac{(n+r-1)!}{(r-1)!(n+r-1-(r-1))} \\
    =\frac{(n+r-1)!}{(n+r-1-(n))!(n)!}
    = \begin{pmatrix} n+r-1 \\ n \end{pmatrix}
\end{align*}
\begin{theorem}
    The number of ways to distribute $n$ identical balls into $r$ distinct boxes is $(\begin{smallmatrix} n+r-1 \\ r-1 \end{smallmatrix})$ or $(\begin{smallmatrix} n+r-1 \\ n \end{smallmatrix})$ ways.
\end{theorem}
\section{Number Theory}
\begin{definition}
    Let $k$ and $n$ be integers with $k \neq 0$. If there exists integer $q$ such that $kq=n$, we say $k$ \emph{divides} $n$, denoted $k \vert n$.
\end{definition}
\begin{theorem}
    Let $a,b,c$ be integers. If $a \vert b$ and $b \vert c$, then $a \vert c$.
\end{theorem}
\begin{theorem}
    Let $a,b,c$ be integers. If $a \vert b$ and $a \vert c$, then $a \vert b+c$.
\end{theorem}
\begin{theorem}
    (The Division Algorithm). Let $m$ and $n$ be integers, with $m>0$. Then, there is a unique pair of integers $q$ and $r$ such that $n = mq+r$ where $0 \leq r < m$. 
\end{theorem}
/*Irrational numbers and $\sqrt{2}$*/
\begin{definition}
    Let $n > 0$ and $p > 1$ be integers. $p$ is \emph{prime} if ``$n\vert p$'' is only true if $n=p$ or $n=1$.
\end{definition}
There is a useful mathematical proof technique called \emph{induction}. Induction operates in two parts. If you want to proove some statement involving an arbitrary $n$ is true, you first proove it is true for the specific case $n=c$ where you choose $c$. Secondly, you show that if the statement holds for $n=k$, then it holds for $n=k+1$. Then, you are done. By the first part, you have prooved the case $n=c$. Then, combining this with the second part of the prood, you know it holds for $n=c+1$,$n=c+2$,$n=\dots$ A formal inductive proof has the following form:
\begin{proof*}
    \emph{Sample Inductive Proof:} \\
    We proceed by the method of Induction, \\
    For the base case of $n=0$, \emph{prove the statement holds for $n=0$} \\
    Now we make the inductive hypotheseis that \emph{the statement holds for arbitrary $k$}. \\
    Now we proceed to proove the inductive step \emph{proove that if statement holds for $k$, it holds for $k+1$}. \\
    Thus by induction \emph{the statement holds}. \qed
\end{proof*}
\section{Logic}
The symbols:
\begin{itemize}
    \item $\lnot$: Represents ``not''. $\lnot A$ is read ``not $A$'' or ``negation of $A$''.
    \item $\land$: Represents ``and''. $A \rightarrow B$ is read ``$A$ and $B$''.
    \item $\lor$: Represents ``or''. $A \lor B$ is read ``$A$ or $B$''.
    \item $\rightarrow$: Represent ``if''. $A \land B$ is read ``$A$ implies $B$'' or ``if $A$ then $B$''.
    \item $\leftrightarrow$: Represent ``necessary and sufficient condition''. $A \leftrightarrow B$ is equivalet to $A \rightarrow B$ and $B \rightarrow A$.
    \item $\forall$: The universal quantifier; represents ``for all''. $\forall x \in D$ is read ``for all $x$ in $D$.''
    \item $\exists$: The existential quantifier; represents ``for some'' or ``there exists''. $\exists x \in D$ is read ``there exists an $x$ in $D$''.
    \item $\exists!$: Represent ``there exists unique''. 
\end{itemize}
\begin{definition}
    A statement that is always false called is a \emph{contradiction}. The classic contradiction is $p \land \lnot p$.
\end{definition}
\begin{definition}
    A statement that is always true is called a \emph{tautology}. The classic tautology is $p \lor \lnot p$.
\end{definition}
\begin{theorem} DeMorgan's Law: \\
    $\lnot (p \land q) \iff \lnot p \lor \lnot q$ and
    $\lnot (p \lor q) \iff \lnot p \land \lnot q$
\end{theorem}
\begin{theorem} General DeMorgan's Law: \\
    $\lnot(p_1 \land p_2 \land \dots \land p_n) \iff (\lnot p_1 \lor \lnot p_2 \lor \dots \lor p_n)$ \\
    $\lnot(p_1 \lor p_2 \lor  \dots \lor  p_n) \iff (\lnot p_1 \land \lnot p_2 \land \dots \land p_n)$ \\
\end{theorem}
\begin{theorem} Reduction ad absurdum,
    $(p \rightarrow q) \iff ((p \land \lnot q) \rightarrow c)$
\end{theorem}
\begin{theorem} Modus ponens,
    $((p \leftarrow q) \land p) \implies q$
\end{theorem}
\begin{theorem} Modus tollens,
    $((p \rightarrow q) \land \lnot q) $
\end{theorem}
\begin{theorem} Law of syllogism,
    $((p \rightarrow q)\land(q \rightarrow r)) \implies (p \rightarrow r)$
\end{theorem}
\begin{theorem} Law of disjunctive syllogism,
    $((p \lor q) \land \lnot p) \implies q$
\end{theorem}
\begin{theorem}
    $\lnot(\forall x \in D, p(x)) \iff \exists x \in D, \lnot p(x)$
\end{theorem}
\section{Set Theory}
There exists objects. Objects can be ``in'', ``belong to'' or be an``element'' of a set. If an object is in a set $S$, we say $a \in S$. If not, we say $a \notin S$. We can describe the elements of a set $S$ in the folliwing notation. $S = \{f \vert \text{rule that $f$ must obey} \}$.
\begin{definition}
    Let $A$ and $B$ be sets. If $\forall x \in A, x \in B$ then $A$ is a \emph{subset} of $B$, denoted $A \subseteq B$.
\end{definition}
\begin{definition}
    The set of everything in the relevant universe is denoted $\mathcal{U}$.
\end{definition}
\begin{definition}
    The set with nothing in it is the \emph{empty set}, denoted $\O$. $\O = \{ f \vert f \notin \mathcal{U} \}$.
\end{definition}
\begin{theorem}
For any set $A$, $\O \subseteq A$.
\end{theorem}
\begin{definition}
Let $A$ and $B$ be sets such that $A \subseteq B$ and $B \subseteq A$. Then, we say $A=B$.
\end{definition}
\begin{definition}
Let $A$ and $B$ be sets. If $A \subseteq B$ but $A \neq B$ then $A$ is a \emph{proper subset} of $B$, denoted $A \subset B$.
\end{definition}
\begin{theorem}
The empty set is unique.
\end{theorem}
\begin{definition}
    Let $n$ be a nonnegative integer. A set containing $n$ distinct elements is called an \emph{n-set}.
\end{definition}
\begin{definition}
    Let $A$ be an n-set. $n$ is called the \emph{cardinality} of $A$ or the \emph{order} of $A$, denoted $\vert A \vert = n$
\end{definition}
\begin{definition}
    Let $A$ and $B$ be sets so that $B \in A$ and $\vert B \vert = r$. Then $B$ is said to be an \emph{r-subset} of $A$.
\end{definition}
\begin{theorem}
    There are $(\begin{smallmatrix} n\\r \end{smallmatrix}$). $r$-subsets of and $n$-set.
\end{theorem}
\begin{definition}
    Let $A$ be a set. The set of all subsets of $A$ is called the \emph{power set of $A$} dentoted $\mathcal{P}(A)$.
\end{definition}
\begin{theorem}
    For any nonnegative integer $n$ there are $2^n$ subsets of an $n$-set $A$. So $\myabs{\mathcal{P}(A)} = 2^n$
\end{theorem}
/*Note about summing over choose operators*/
\begin{definition}
    Let $A$ and $B$ be sets. The \emph{union} of $A$ and $B$ denoted $A \cup B$ is defined $A \cup B = \{ x \vert x \in A \text{ or } x \in B \}$.
\end{definition}
\begin{definition}
    Let $A$ and $B$ be sets. The \emph{intersection} of $A$ and $B$ denoted $A \cap B$ is defined $A \cap B = \{ x \vert x \in A \text{ and } x \in B \}$.
\end{definition}
\begin{definition}
    Let $A$ and $B$ be sets. If $A \cap B = \O$ then $A$ and $B$ are \emph{disjoint}.
\end{definition}
\begin{theorem}
    Let $A$ and $B$ be sets. Then $\myabs{A \cup B} = \myabs{A} + \myabs{B} - \myabs{A \cap B}$.
\end{theorem}
\begin{theorem} 
    Let $A$ and $B$ be sets \\ Associative laws:\\
    $(A \cup B) \cup C = A \cup (B \cup C)$ and 
    $(A \cap B) \cap C = A \cap (B \cap C)$ \\
    Distributive laws:\\
    $(A \cup B) \cap C = (A \cap C) \cup (B \cap C)$ and
    $(A \cap B) \cap C = (A \cup C) \cap (B \cup C)$
\end{theorem}
\begin{definition}
    Let $A$ and $B$ be sets. The \emph{relative complement} of $B$ \emph{in} $A$, denoted $A \setminus B$ or $A - B$. Its defined as $A \setminus B = \{ x \vert x \in A \text{ and } x \notin B \}$.
\end{definition}
\begin{definition}
    Let $A$ be a set. The \emph{complement} of $A$, denoted $\overline{A}$ is defined as $\overline{A} = \{ x \in \mathcal{U} \vert x \notin A \}$.
\end{definition}
\begin{definition} (DeMorgan's Laws) Let $A$ and $B$ be sets. \\
    Then $\overline{A \cup B} = \overline{A} \cap \overline{B}$ and $\overline{A \cap B} = \overline{A} \cup \overline{B}$
\end{definition}
\end{document}
