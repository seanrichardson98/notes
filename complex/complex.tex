\documentclass[11pt]{article}

\usepackage{amsmath}
\usepackage{amsthm}
\usepackage{amsfonts}

\usepackage{mathtools}

\usepackage{float}
\usepackage{enumitem}
\usepackage{commath}
\usepackage{tikz}
\usetikzlibrary{calc}
\usepackage{cancel}
\usepackage{subfigure}
\usepackage{multicol}

\usepackage{graphicx}
\graphicspath{{images/}}

\usepackage{my_notes}
\usepackage{my_math}

\theoremstyle{definition}
\newtheorem{theorem}{Theorem}[section]
\newtheorem{prop}[theorem]{Proposition}
\newtheorem{definition}[theorem]{Definition}
\newtheorem{corollary}[theorem]{Corollary}
\newtheorem{ex}[theorem]{Example}

\newenvironment{just}{\textit{Justification.}}{}
\newenvironment{trans}{\textit{Translation.}}{}

\DeclareMathOperator{\re}{Re}
\DeclareMathOperator{\im}{Im}

\setlength\parindent{0pt}

\newcommand{\pder}[2]{\frac{\partial{#1}}{\partial{#2}}}

\begin{document}

\title{Complex Analysis}
\author{Sean Richardson}
\date{\today}
\maketitle

\section{Basics}

\subsection{Definitions}
\begin{definition}[Complex Number]
We define a complex number $z$ to be of the form $z=x+iy$. Where $x,y \in
\mathbb{R}$ and $i$ is defined such that $i^2=-1$. We say $z \in
\mathbb{C}$.
\end{definition}
\begin{definition}[Conjugate]
    If $z \in \mathbb{C}$ such that $x = x+iy$ then $z$ \emph{conjugate}, written $\bar{z}$, is
    defined $\bar{z} = x-iy$.
\end{definition}
\begin{definition} Some notation definitions
    \begin{itemize}
        \item $\myabs{z}^2 \coloneqq  z\bar{z}$
        \item If $z = x+iy$ then $\re(z) \coloneqq x$.
        \item If $z = x+iy$ then $\im(z) \coloneqq y$.
        \item /*$\arg(z) = \theta$*/
    \end{itemize}
\end{definition}
/*Complex Plane and def reps*/
\subsection{Properties}
\begin{prop} Take $z_1,z_2 \in \mathbb{C}$
    \begin{itemize}
        \item $\re(z_1+z_2) = \re(z_1)+\re(z_2)$
        \item $\im(z_1+z_2) = \im(z_1)+\im(z_2)$
        \item $\overline{z_1+z_2} = \bar{z_1}+\bar{z_2}$
        \item $\overline{z_1 z_2} = \bar{z_1}\bar{z_2}$
        \item $\myabs{z_1 z_2} = \myabs{z_1}\myabs{z_2}$
        \item $\arg(z_1 z_2) = \arg(z_1) + \arg(z_2)$
    \end{itemize}
\end{prop}
\begin{prop}[Triangle Inequality]
    $\myabs{z_1+z_2} \leq \myabs{z_1} + \myabs{z_2}$
\end{prop}
\begin{prop}[/*?*/]
    $e^{i\theta} = \cos{\theta} + i\sin{\theta}$
\end{prop}
\begin{prop}
    $z^n = \myabs{z}^n e^{in\theta}$
\end{prop}
\begin{prop}
    $\sqrt[n]{z}$ is anything such that ${(\sqrt[n]{z})}^n=z$.
    So,
    $$\sqrt[n]{z}=\{ \sqrt[n]{\myabs{z}}e^{\frac{i(\theta+2\pi k)}{n}}
    \mid 0 \leq k < n \}$$
\end{prop}

\section{Transformations}
\subsection{Intro}
\begin{ex}~
    \begin{itemize}
    \item $R(z) = e^{i\alpha}z$ defines a rotation by $\alpha$
    \item $T(z) = z+z_0$ translates $0$ to $z_0$
    \item $S(z) = \alpha z$ scales/dilates the plane by $\alpha$ centered
        at $0$
    \end{itemize}
\end{ex}
\begin{ex}
    $z \to z^2$ transformation /*explanation and visual*/
\end{ex}
\subsection{Continuity}
\begin{definition}[Continuity]
    If we have some function $f: \mathbb{C} \to \mathbb{C}$. We say $f$ is
    continuous at $z \in \mathbb{C}$ if for every sequence $\{ z_k \}$, we
    have $z_k \to z \implies f(z_k) \to f(z)$. Or, intuitively, if $z
    \approx a$, we have $f(z) \approx f(a)$.
\end{definition}

\subsection{Branches}
If we have some function $f$ that maps from one value to a set of values,
$f: \mathbb{C} \to \{z \in \mathbb{C}\}$. Then, we want to restrict the
function such that $f$ is $1-1$ and continuous.
\begin{definition}
    If we have a function $f: \mathbb{C} \to {z \in \mathbb{C}}$
\end{definition}
/*Principle Branch*/\\
/*Riemann Surface*/

\end{document}










