\documentclass[11pt]{article}

\usepackage{amsmath}
\usepackage{amsthm}
\usepackage{amsfonts}

\usepackage{float}
\usepackage{enumitem}
\usepackage{commath}
\usepackage{tikz}
\usetikzlibrary{calc}
\usepackage{cancel}
\usepackage{subfigure}
\usepackage{multicol}

\usepackage{graphicx}
\graphicspath{{images/}}

\usepackage{my_notes}
\usepackage{my_math}

\theoremstyle{definition}
\newtheorem{theorem}{Theorem}[section]
\newtheorem{definition}[theorem]{Definition}
\newtheorem{corollary}[theorem]{Corollary}

\newenvironment{just}{\textit{Justification.}}{}
\newenvironment{trans}{\textit{Translation.}}{}

\setlength\parindent{0pt}

\begin{document}
\title{Calculus III}
\author{Sean Richardson}
\date{\today}
\maketitle
\tableofcontents

\section{2D and 3D space}
\subsection{Cartesian Coordinates}
\subsection{Contour Maps}
\subsection{Quadratic Surfaces}
\begin{align}
    f(x,y) &= x^2+y^2
    \tag{Paraboloid}\\
    f(x,y) &= x^2-y^2
    \tag{Saddle}\\
    f(x,y) &= xy
    \tag{Saddle}
\end{align}
\begin{theorem}
    An equation of the form $f(x,y) = ax^2+bxy+cy^2$ is a saddle if
    $b^2-4ac > 0$ and a bowl if $b^2-4ac <0 $.
\end{theorem}
\subsection{Vector Fields}
\begin{definition}[Vector Field]
    A vector field is some mapping $V: S \to \mathbb{R}^n$ with $S \subset
    \mathbb{R}^n$. Visually, we represent vector fields by drawing the
    vector of the output centered at the input point.
    /*Visual vector field*/
\end{definition}
\subsection{Transformations}
\begin{definition}[Transformation]
    A transformation is a mapping $T:\mathbb{R}^n\to\mathbb{R}^m$. We
    represent this mapping $\mathbf{x} \to \mathbf{y}$ with an equation of the form 
    $$(y^1,\dots,y^m) = T(x^1,\dots,x^n)=(T_1(x^i),\dots,T_m(x^i))$$
\end{definition}
Note that \emph{linear-ish} transformations can be written 
\begin{align*}
 \begin{pmatrix} y^1 \\ \vdots \\ y^m \end{pmatrix}
 &=\begin{pmatrix} T_1(x^1,\dots,x^n) \\ \vdots \\ T_m(x^1,\dots,x^n) \end{pmatrix}
 = \begin{pmatrix} b^1 + \sum a^1_i x^i \\ \vdots \\ b^m + \sum a^m_i x^i \end{pmatrix}
 = \begin{pmatrix} b^1 \\ \vdots \\ b^m \end{pmatrix}
 + \sum{\begin{pmatrix} a^1_i \\ \vdots \\ a^m_i \end{pmatrix}x^i} \\
 &= \begin{pmatrix} b^1 \\ \vdots \\ b^m \end{pmatrix}
 + \begin{pmatrix} a^1_1 & \dots & a^1_n \\ \vdots & \ddots & \vdots \\
 a^m_1 & \dots & a^m_n \end{pmatrix}
 \begin{pmatrix} x_1 \\ \vdots \\ x_n \end{pmatrix}
\end{align*}

\subsection{Vector Products}
\begin{theorem}
    For vectors $\mathbf{u},\mathbf{v} \in \mathbb{R}^n$ separated by some
    angle $\theta$, we have
    \begin{equation*}
        \sum{u^i v^i} = \mymag{\mathbf{u}}\mymag{\mathbf{v}}\cos(\theta)
    \end{equation*}
\end{theorem}
\begin{definition}[Dot Product]
    The previous equivalency is quite useful, so we define the \emph{dot
    product} by the mapping $\boldsymbol{\cdot} : \mathbb{R}^n \times
    \mathbb{R}^n \to \mathbb{R}$ such that
    \begin{equation*}
        u \cdot v = \sum{u^i v^i} = \mymag{\mathbf{u}}\mymag{\mathbf{v}}\cos(\theta)
    \end{equation*}
    Where $\mathbf{u},\mathbf{v} \in \mathbb{R}^n$ separated by angle
    $\theta$.
\end{definition}
\begin{definition}[Cross Product]
    The \emph{cross product} is some mapping $\times : \mathbb{R}^3 \times
    \mathbb{R}^3 \to \mathbb{R}^3$ such that for $\mathbf{a,b} \in
    \mathbb{R}^3$, $a = \myvec{a_x,a_y,a_z}$ and $v = \myvec{b_x,b_y,b_z}$
    we have
    \begin{equation*}
        a \times b = \myvec{(a_y b_z - a_z b_y), (a_z b_x - a_x b_z), (a_x
        b_y - a_y b_x)}.
    \end{equation*}
\end{definition}
By abuse of notation, we can write the cross product as:
\begin{equation*}
    a \times b = \det
    \begin{pmatrix} 
    \mathbf{i}       & \mathbf{j}   & \mathbf{k}   \\
    a_x              & a_y          & a_z          \\
    b_x              & b_y          & b_z          \\
    \end{pmatrix}
\end{equation*}

\section{???}
\subsection{Curvilinear Coordinates}
\subsubsection{Polar Coordinates}
\begin{definition}
Polar coordinates are defined by a transformation from $(r, \theta)$ space
to $(x,y)$ space. \\
Below are transformations $T: (r, \theta) \to (x,y)$ and $T^{-1}: (x,y) \to
(r,\theta)$
\begin{align*}
    T&: (x,y) = (r\cos(\theta),r\sin(\theta)) \\
    T^{-1}&: (r,\theta) = (\sqrt{x^2+y^2},\arctan(y/x)[+\pi?])
\end{align*}
Note: In $T^{-1}, $
\end{definition}
\subsubsection{Cylindrical Coordinates}
\subsubsection{Spherical Coordinates}
\end{document}
