\documentclass[11pt]{article}

\usepackage{amsmath}
\usepackage{amsthm}
\usepackage{amsfonts}

\usepackage{float}
\usepackage{enumitem}
\usepackage{commath}
\usepackage{tikz}
\usetikzlibrary{calc}
\usepackage{cancel}
\usepackage{subfigure}
\usepackage{multicol}

\usepackage{graphicx}
\graphicspath{{images/}}

\usepackage{my_notes}
\usepackage{my_math}

\theoremstyle{definition}
\newtheorem{theorem}{Theorem}[section]
\newtheorem{definition}[theorem]{Definition}
\newtheorem{corollary}[theorem]{Corollary}

\newenvironment{just}{\textit{Justification.}}{}
\newenvironment{trans}{\textit{Translation.}}{}

\setlength\parindent{0pt}

\newcommand{\pder}[2]{\frac{\partial{#1}}{\partial{#2}}}

\begin{document}
\title{Calculus III}
\author{Sean Richardson}
\date{\today}
\maketitle
\tableofcontents

\section{2D and 3D space}
\subsection{Cartesian Coordinates}
\subsection{Contour Maps}
\subsection{Quadratic Surfaces}
\begin{align}
    f(x,y) &= x^2+y^2
    \tag{Paraboloid}\\
    f(x,y) &= x^2-y^2
    \tag{Saddle}\\
    f(x,y) &= xy
    \tag{Saddle}
\end{align}
\begin{theorem}
    An equation of the form $f(x,y) = ax^2+bxy+cy^2$ is a saddle if
    $b^2-4ac > 0$ and a bowl if $b^2-4ac <0 $.
\end{theorem}
\subsection{Vector Fields}
\begin{definition}[Vector Field]
    A vector field is some mapping $V: S \to \mathbb{R}^n$ with $S \subset
    \mathbb{R}^n$. Visually, we represent vector fields by drawing the
    vector of the output centered at the input point.
    /*Visual vector field*/
\end{definition}
\subsection{Transformations}
\begin{definition}[Transformation]
    A transformation is a mapping $T:\mathbb{R}^n\to\mathbb{R}^m$. We
    represent this mapping $\mathbf{x} \to \mathbf{y}$ with an equation of the form 
    $$(y^1,\dots,y^m) = T(x^1,\dots,x^n)=(T_1(x^i),\dots,T_m(x^i))$$
\end{definition}
Note that \emph{linear-ish} transformations can be written 
\begin{align*}
 \begin{pmatrix} y^1 \\ \vdots \\ y^m \end{pmatrix}
 &=\begin{pmatrix} T_1(x^1,\dots,x^n) \\ \vdots \\ T_m(x^1,\dots,x^n) \end{pmatrix}
 = \begin{pmatrix} b^1 + \sum a^1_i x^i \\ \vdots \\ b^m + \sum a^m_i x^i \end{pmatrix}
 = \begin{pmatrix} b^1 \\ \vdots \\ b^m \end{pmatrix}
 + \sum{\begin{pmatrix} a^1_i \\ \vdots \\ a^m_i \end{pmatrix}x^i} \\
 &= \begin{pmatrix} b^1 \\ \vdots \\ b^m \end{pmatrix}
 + \begin{pmatrix} a^1_1 & \dots & a^1_n \\ \vdots & \ddots & \vdots \\
 a^m_1 & \dots & a^m_n \end{pmatrix}
 \begin{pmatrix} x_1 \\ \vdots \\ x_n \end{pmatrix}
\end{align*}

\subsection{Projection}
\begin{theorem}
    $$\mathbf{a_H} = \left(\frac{\mathbf{a}
\boldsymbol{\cdot}\mathbf{b}}{\mymag{\mathbf{b}}^2}\right)\mathbf{b}$$
\end{theorem}
\subsection{Vector Products}
\begin{theorem}
    For vectors $\mathbf{u},\mathbf{v} \in \mathbb{R}^n$ separated by some
    angle $\theta$, we have
    \begin{equation*}
        \sum{u^i v^i} = \mymag{\mathbf{u}}\mymag{\mathbf{v}}\cos(\theta)
    \end{equation*}
\end{theorem}
\begin{definition}[Dot Product]
    The previous equivalency is quite useful, so we define the \emph{dot
    product} by the mapping $\boldsymbol{\cdot} : \mathbb{R}^n \times
    \mathbb{R}^n \to \mathbb{R}$ such that
    \begin{equation*}
        u \cdot v = \sum{u^i v^i} = \mymag{\mathbf{u}}\mymag{\mathbf{v}}\cos(\theta)
    \end{equation*}
    Where $\mathbf{u},\mathbf{v} \in \mathbb{R}^n$ separated by angle
    $\theta$.
\end{definition}
\begin{definition}[Cross Product]
    The \emph{cross product} is some mapping $\times : \mathbb{R}^3 \times
    \mathbb{R}^3 \to \mathbb{R}^3$ such that for $\mathbf{a,b} \in
    \mathbb{R}^3$, $a = \myvec{a_x,a_y,a_z}$ and $v = \myvec{b_x,b_y,b_z}$
    we have
    \begin{equation*}
        a \times b = \myvec{(a_y b_z - a_z b_y), (a_z b_x - a_x b_z), (a_x
        b_y - a_y b_x)}.
    \end{equation*}
\end{definition}
By abuse of notation, we can write the cross product as:
\begin{equation*}
    a \times b = \det
    \begin{pmatrix} 
    \mathbf{i}       & \mathbf{j}   & \mathbf{k}   \\
    a_x              & a_y          & a_z          \\
    b_x              & b_y          & b_z          \\
    \end{pmatrix}
\end{equation*}
Note that:
\begin{itemize}
    \item $\mymag{\mathbf{a} \times \mathbf{b}} = \text{Area of
parallelogram spanned by } \mathbf{a} \text{ and } \mathbf{b}$.
    \item $\mathbf{a} \times \mathbf{b}$ point in a direction
        orthogonal to both $\mathbf{a}$ and $\mathbf{b}$.
\end{itemize}
\subsection{Volumes}
\begin{theorem}[Volumes of Parallelotopes]
Let $V(\mathbf{x}^1,\dots,\mathbf{x}^n) = $ Volume of the paralleletope
spanned by $\mathbf{x}^1,\dots,\mathbf{x}^n$.
Then,
$$V(\mathbf{x}^1,\dots,\mathbf{x}^n) = \sqrt{\det
\begin{pmatrix}
    \mathbf{x}^1 \cdot \mathbf{x}^1 & \cdots & \mathbf{x}^1 \cdot
    \mathbf{x}^n \\
    \vdots & \ddots & \vdots \\
    \mathbf{x}^n \cdot \mathbf{x}^1 & \cdots & \mathbf{x}^n \cdot
    \mathbf{x}^n \\
\end{pmatrix}}
$$
Note if we call the matrix $g$ we have $g_{ij} = \mathbf{x^i} \cdot
\mathbf{x^j}$.\\
If we are working in $\mathbb{R}^n$ and $\mathbf{x^i} = \langle x^i_1,
\dots, x^i_n \rangle$ then,
$$V(\mathbf{x}^1,\dots,\mathbf{x}^n) = \left\vert\det
\begin{pmatrix}
    x_1^1  & \cdots & x_n^1 \\
    \vdots & \ddots & \vdots \\
    x_1^n  & \cdots & x_n^n \\


\end{pmatrix}\right\vert
$$

\end{theorem}
\section{???}
\subsection{Curvilinear Coordinates}
\subsubsection{Polar Coordinates}
\begin{definition}[Polar Coordinates]
Polar coordinates are defined by a transformation from $(r, \theta)$ space
to $(x,y)$ space. \\
Below are transformations $T: (r, \theta) \to (x,y)$ and $T^{-1}: (x,y) \to
(r,\theta)$
\begin{align*}
    T&: (x,y) = (r\cos(\theta),r\sin(\theta)) \\
    T^{-1}&: (r,\theta) = (\sqrt{x^2+y^2},\arctan(y/x)[+\pi?])
\end{align*}
Note: In $T^{-1}$ you must add $\pi$ radians to $\theta$ if $x<0$. 

\end{definition}
\subsubsection{Cylindrical Coordinates}
\begin{definition}[Cylindrical Coordinates]
Cylindrical coordinates are defined through a transformation from $(r,
\theta, z)$ space to $(x,y,z)$ space. 
Below are transformations $T: (r, \theta,z) \to (x,y,z)$ and $T^{-1}:
(x,y,z) \to (r,\theta,z)$
\begin{align*}
    T&: (x,y,z) = (r\cos(\theta),r\sin(\theta),z) \\
    T^{-1}&: (r,\theta,z) = (\sqrt{x^2+y^2},\arctan(y/x)[+\pi?],z)
\end{align*}
Note: In $T^{-1}$ you must add $\pi$ radians to $\theta$ if $x<0$. 
\end{definition}
\subsubsection{Spherical Coordinates}
Spherical coordinates are defined through a transformation from
$(r,\theta,\phi)$ space to $(x,y,z)$ space.
Below are transformations $T: (r, \theta, \phi) \to (x,y,z)$ and $T^{-1}:
(x,y,z) \to (r,\theta, \phi)$
\begin{align*}
    T&: (x,y,z) =
    (r\cos\theta\sin\phi,r\cos\theta\sin\phi,r\cos\phi)\\
    T^{-1}&: (x,y,z) =
    (\sqrt{x^2+y^2+z^2},\arctan(y/x)[+\pi?],\arccos(\frac{z}{\sqrt{x^2+y^2+z^2}}))
\end{align*}
\subsection{Partial Vectors}
\begin{definition}[Partial Vector Field]
Consider some transformation $T: \mathbb{R}^n \to \mathbb{R}^m$ given by
$(y^1,\dots,y^m) = T(x^1, \dots, x^n)$. We then define the partial vector
field for some coordinate $x^i$ in $y$ space, denoted $\partial_{x^i}$, by
$$ \partial_{x_i} = \left\langle \pder{y^1}{x^i}, \dots, \pder{y^m}{x^i} \right\rangle$$
Note that $\partial_{x_i}$ points in the direction of increase of $x_i$ at
any location $\mathbf{y}$.
\end{definition}
\subsection{Transformation Approximation}
\subsubsection{Linear}
\begin{definition}[Jacobi Matrix]
    Consider some transformation $T: \mathbb{R}^n \to \mathbb{R}^m$ given by
    $(y^1,\dots,y^n) = T(x^1, \dots, x^m)$.
    We define the Jacobi Matrix $DT$ to be an $n \times m$ matrix such that
    $DT_{ij} = \pder{y^j}{x^i}$ Or,
    $$DT =
    \begin{pmatrix}
    \pder{y^1}{x^1} & \cdots & \pder{y^1}{x^n}\\
    \vdots          & \ddots & \vdots \\
    \pder{y^m}{x^1} & \cdots & \pder{y^m}{x^n}
    \end{pmatrix}
    $$
\end{definition}
\begin{theorem}
    If we evaluate $DT$ at a specific point $\mathbf{p}$ in $x$ space,
    denoted $DT \vert_{\mathbf{p}}$, then this matrix is a linear
    approximation of $T$ at $\mathbf{p}$. So if 
    $\mathbf{\Delta x} = \mathbf{x} - \mathbf{p}$ 
    and $\mathbf{\Delta y} = T(\mathbf{x}) - T(\mathbf{p})$ then,
    $$
    \mathbf{\Delta y} \approx DT\vert_{\mathbf{p}} \cdot \mathbf{\Delta x}
    $$
\end{theorem}
\subsubsection{Quadratic}
\begin{definition}[Hessian Matrix]
    Consider some transformation $f: \mathbb{R}^n \to \mathbb{R}$ given by 
    $z = f(x^1,\dots,x^n)$. Then, we define the Hessian Matrix $Hessf$ to
    an $n \times n$ matrix given by $Hessf_{ij} = \pder{f}{x^ix^j}$.
    Or,
    $$
    Hessf =
    \begin{pmatrix}
        \pder{f}{x^1x^1} & \dots  & \pder{f}{x^1x^n} \\
        \vdots           & \ddots & \vdots           \\
        \pder{f}{x^nx^1} & \cdots & \pder{f}{x^nx^n}  
    \end{pmatrix}
    $$
\end{definition}
\begin{theorem}[Quadratic Approximation]
    Take some point $\mathbf{p}$ in $x$ space to center our approximation
    about. Then if we take $\mathbf{\Delta x} = \mathbf{x} - \mathbf{p}$
    and $\Delta y = f(\mathbf{x}) - f(\mathbf{p})$ we have,
    $$
    \Delta y \approx Df\vert_{\mathbf{p}} \cdot \mathbf{\Delta x}
    + \frac{1}{2} \cdot \mathbf{{\Delta x}^T} \cdot Hessf\vert_{\mathbf{p}}
    \cdot \mathbf{\Delta x}
    $$
\end{theorem}
\subsection{Chain Rule}

\subsection{Optimization}

\section{Calculus}
\subsection{Line Integral}
\begin{theorem}
    Let $\mathcal{L} = (x^1(t), \dots, x^n(t))$ describe some line in
    $\mathbb{R}^n$. Then, the length of the line between between $t=t_i$
    and $t=t_f$ is given by
    $$\int_{t_i}^{t_f} \mymag{\partial_t}dt$$
    If we assign a density $\rho$
\end{theorem}
\end{document}
