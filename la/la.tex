\documentclass[11pt]{article}

\usepackage{amsmath}
\usepackage{amsthm}
\usepackage{amsfonts}

\usepackage{float}
\usepackage{enumitem}
\usepackage{commath}
\usepackage{tikz}
\usetikzlibrary{calc}
\usepackage{cancel}
\usepackage{subfigure}
\usepackage{multicol}

\usepackage{graphicx}
\graphicspath{{images/}}

\usepackage{my_notes}
\usepackage{my_math}

\theoremstyle{definition}
\newtheorem{theorem}{Theorem}[section]
\newtheorem{definition}[theorem]{Definition}
\newtheorem{corollary}[theorem]{Corollary}

\newenvironment{just}{\textit{Justification.}}{}
\newenvironment{trans}{\textit{Translation.}}{}

\setlength\parindent{0pt}

\newcommand{\pder}[2]{\frac{\partial{#1}}{\partial{#2}}}
\newcommand{\der}[2]{\frac{d{#1}}{d{#2}}}

\begin{document}
\title{Linear Algebra}
\author{Sean Richardson}
\date{\today}
\maketitle

\section{}
\begin{definition}[Linear Equation]

\end{definition}
\begin{definition}[Linear Combination]

\end{definition}
\begin{definition}[Gaussian Elmination]

\end{definition}
\begin{definition}[Basis]

\end{definition}
\begin{definition}[Subspace]

\end{definition}
\begin{definition}[Span]

\end{definition}
\begin{definition}[System of Equation]

\end{definition}
\begin{definition}[Linear Dependence/Independence]

\end{definition}

\section{}
\begin{definition}[Inverse]
    Take some matrix $A$. Then, the i
\end{definition}

\end{document}
